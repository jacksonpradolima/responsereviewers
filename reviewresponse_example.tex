% !TeX encoding = UTF-8
% !TeX spellcheck = en_US

% Copyright Javier Sánchez-Monedero.
% Please report bugs and suggestions to (jsanchezm at uco.es)
%
% This document is released under a Creative Commons Licence 
% CC-BY-SA (http://creativecommons.org/licenses/by-sa/3.0/) 
%
% BASIC INSTRUCTIONS: 
% 1. Load and set up proper language packages
% 2. Complete the paper data commands
% 3. Use commands \rcomment and \newtext as shown in the example

% Document with colored options
\documentclass[a4paper,twoside,10pt,usenames,dvipsnames]{reviewresponse}

% 1. Load and set up proper language packages
\usepackage[english]{babel}
\usepackage{caption,subcaption}

\usepackage{lipsum} % Fake text

% 2. Complete the paper data
\newcommand{\myAuthors}{{John.~Doe$^{\displaystyle 1}$, ~Foo~B.~Bar$^{\displaystyle 2}$, } \\ {~Peter Sanchez$^{\displaystyle 2}$}}
\newcommand{\myAuthorsShort}{John.~Doe et. al}
\newcommand{\myEmails}{john@uco.es,abc@def.com}
\newcommand{\manuscriptID}{Response Letter for the Manuscript XXX-XXXX-XX-XXXX}
\newcommand{\myTitle}{My Submited Paper Tittle}
%\newcommand{\myShortTitle}{Response to reviewers}
\newcommand{\myJournal}{Journal Name}
\newcommand{\myDept}{{$^{\displaystyle 1}$Department of Computer Science and Numerical Analysis, University of Córdoba, Córdoba 14074, Spain. \\ \url{http://www.uco.es/ayrna/}}\\
{$^{\displaystyle 2}$School of Computer Science, The University of XXX. }\\}

\hypersettings

%%%%%%%%%%%%%%%%%%%%%%%%%%%%%%%%%%%%%%%%%%%%%%%%%%%%%%%%%%%%%%%%%%%%%%%%%%

\begin{document}

\thispagestyle{plain}

\header

\noindent We thank the anonymous Reviewers for their efforts and feedback.

\tableofcontents

%%%%%%%%%%%%%%%%%%%%%%%%%%%%%%%%%%%%%%%%%%%%%%%%%%%%%%%%%%%%%%%%%%%%%%%%%%%%

\listoftodos[Notes]

%%%%%%%%%%%%%%%%%%%%%%%%%%%%%%%%%%%%%%%%%%%%%%%%%%%%%%%%%%%%%%%%%%%%%%%%%%%%

\section*{Comments from the Associate Editor}
%%%%%%
% This part below is required
\label{sec:editor}
\addcontentsline{toc}{section}{\nameref{sec:editor}}
\markboth{Comments from the Associate Editor}{Comments from the Associate Editor}
%%%%%%

\ecomment{
	\lipsum[2-4]
}

\response{
	We revised many parts to address the main points mentioned by the referees. 
}


%%%%%%%%%%%%%%%%%%%%%%%%%%%%%%%%%%%%%%%%%%%%%%%%%%%%%%%%%%%%%%%%%%%%%%%%%%%%

\section{Comments from Reviewer \#1}

\rcomment{
This work proposed a new method to address...
}

\response{
We thank the Reviewer for the positive comments about our work and manuscript. Below, we address every comment carefully and explain the corresponding changes in the manuscript.
}

\rcomment{
Nevertheless, I personally do not agree with some of the views of the authors...
}

\response{
We thank the reviewer for this comment and we agree with him/her. We have rewritten the paragraph following his/her comments:  


\mention{
Lorem Ipsum is simply dummy text of the printing and typesetting industry. Lorem Ipsum has been the industry's standard dummy text ever since the 1500s, when an \newtext{unknown printer} took a galley of type and scrambled it to make a type specimen book. It has survived not only five centuries, but also the leap into electronic typesetting, remaining essentially unchanged. It was popularised in the 1960s \newtext{the release of Letraset sheets containing Lorem Ipsum passages}, and more recently with desktop publishing software like Aldus PageMaker including versions of Lorem Ipsum.
}

}

\clearpage

\section{Comments from Reviewer \#2}

\rcomment
{
I would like to see a few add ons to substantiate the approach in particular regarding its limits...
}

\response{
We want to thank the reviewer for this comment. In order to satisfy this petition we have added a new \hl{Subsection}...
\unsure[inline]{We need to answer this question}
}

% Uncomment in case references are needed
%\bibliographystyle{apalike}
%\bibliography{responsereferences}


\end{document}
